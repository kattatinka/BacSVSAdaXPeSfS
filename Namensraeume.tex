Ein Namensraum ist ein Unterscheidungsmerkmal, das ein Element oder Attribut eindeutig bestimmt. Hierzu wird meist eine URI-Referenz, die von den meisten XML-Parsern unterstützt wird, verwendet. Diese dient ledeglich der eindeutigen Kennzeichnung, der XML-Parser sucht nicht nach einer konkret hinterlegten Namensliste. Durch Namensräume können verschiedene Module und Anwendungen auf dasselbe XML-Dokument zugreifen und Informationen erhalten ohne das es zu Konflikten bei Namensüberschneidungen kommt. 


%Java und xml d

%Namensräume sind ein wichtiges Konzept bei der Verarbeitung von xml doks. Werden von meisten xml parsern unterstützt. Kann muss aber nicht.
%In modernen XML-Anwendungen kann es vorkommern, dass in ein und demselben XML-Dok Informationen für verschiedene Module einer Anwendung oder sogar verschiedenn Anwengungen ennthalten sind. Edes Modul oder jede Anwendung ist dabei nur an bestimmten Infor im Dok interessiert.

%Um Unterscheidungen der xml inhalte zu vereinfachen, wurden Namensräume eingeführt.
%Ein Namensraum ist ein Unterscheidungsmerkmal, das Elementen oder Attributen zusätzlich zu ihrem Namen gegeben werden kann. Namensüberschneidungen werden  möglich ohne dass konflikte entstehen. Das zusätzliche Merkmal ist visuell im Dokument sichtbar, aber auch für den Parser  Ein Namensraum wird eindeutig durch einen URI identifiziert. Der Hintergedanke besteht  darin, dass die Domainnamen weltweit eindeutig sind.

%Der Namensraum wird über ein sogenanntes Präfix einem Element oder Attribut zugeordnet.

%%%%%%%%%%%%%%%%%%%%%%%%%%%%%%%%%%%%%%%%%%%%%%%%%%%%%%%%%%%%%%%

%java und jml e

%namespaces allows to create xml data that merges multiple vocabularies within a single dok type.
%prefix the element context to each attribute name
%muss nicht wirklich existieren
%default namespaces

%%%%%%%%%%%%%%%%%%%%%%%%%%%%%%%%%%%%%%%%%%%%%%%%%

%w3c http://www.schumacher-netz.de/TR/1999/REC-xml-names-19990114-de.html

%Wir betrachten Anwendungen der Extensible Markup Language (XML), in denen ein einzelnes XML- Dokument Elemente und Attribute (hier "Markup-Vokabular" genannt) enthalten kann, die für verschiedene Software-Module definiert sind und von verschiedenen Software-Modulen verwendet werden. Eine Motivation dafür ist Modularität; wenn ein Markup-Vokabular existiert, das gut verstanden wird und für welches nützliche Software vorhanden ist, ist es besser, dieses Markup wieder zu verwenden als es neu zu erfinden.

%%%%%%%%%%%%%%%%%%%%%%%%%%%%%%%%%%%%%%%%%%%%%%%%5

%http://www.itwissen.info/definition/lexikon/Namensraum-XML-namespace-XML.html 17.okt 13:45

%Namensräume (engl. name spaces) sind in XML-Dokumenten ein einfaches Verfahren zur eindeutigen Benennung von Element- und Attributnamen. Dabei werden die Element- und Attributnamen so verknüpft, dass diese durch URI-Referenzen eindeutig identifiziert werden können. Diesbezüglich ist die URI-Referenz lediglich eine eindeutige Kennzeichnung, der XML-Prozessor oder auch Parser sucht anders als bei einer Referenz auf eine Document Type Definition (DTD) oder ein XML-Schema keine dort konkret hinterlegte Namensliste. 
