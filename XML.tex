\section{XML}

\subsection{Grundlagen}

markup language /Beschreibungssprache

Einsatzgebiete

Namespaces um das zusammenzulegen und alles passt zusammen

"wellformedness" wird vom parser überprüft

web soap

generelle funktion von xml parsern und alles

serialisierung und validierung


Xml is standard, you van generate data documents that may be processed by anyone else who uses xml 

\subsection{Namensräume}

Ein Namensraum ist ein Unterscheidungsmerkmal, das ein Element oder Attribut eindeutig bestimmt. Hierzu wird meist eine URI-Referenz, die von den meisten XML-Parsern unterstützt wird, verwendet. Diese dient ledeglich der eindeutigen Kennzeichnung, der XML-Parser sucht nicht nach einer konkret hinterlegten Namensliste. Durch Namensräume können verschiedene Module und Anwendungen auf dasselbe XML-Dokument zugreifen und Informationen erhalten ohne das es zu Konflikten bei Namensüberschneidungen kommt. 


%Java und xml d

%Namensräume sind ein wichtiges Konzept bei der Verarbeitung von xml doks. Werden von meisten xml parsern unterstützt. Kann muss aber nicht.
%In modernen XML-Anwendungen kann es vorkommern, dass in ein und demselben XML-Dok Informationen für verschiedene Module einer Anwendung oder sogar verschiedenn Anwengungen ennthalten sind. Edes Modul oder jede Anwendung ist dabei nur an bestimmten Infor im Dok interessiert.

%Um Unterscheidungen der xml inhalte zu vereinfachen, wurden Namensräume eingeführt.
%Ein Namensraum ist ein Unterscheidungsmerkmal, das Elementen oder Attributen zusätzlich zu ihrem Namen gegeben werden kann. Namensüberschneidungen werden  möglich ohne dass konflikte entstehen. Das zusätzliche Merkmal ist visuell im Dokument sichtbar, aber auch für den Parser  Ein Namensraum wird eindeutig durch einen URI identifiziert. Der Hintergedanke besteht  darin, dass die Domainnamen weltweit eindeutig sind.

%Der Namensraum wird über ein sogenanntes Präfix einem Element oder Attribut zugeordnet.

%%%%%%%%%%%%%%%%%%%%%%%%%%%%%%%%%%%%%%%%%%%%%%%%%%%%%%%%%%%%%%%

%java und jml e

%namespaces allows to create xml data that merges multiple vocabularies within a single dok type.
%prefix the element context to each attribute name
%muss nicht wirklich existieren
%default namespaces

%%%%%%%%%%%%%%%%%%%%%%%%%%%%%%%%%%%%%%%%%%%%%%%%%

%w3c http://www.schumacher-netz.de/TR/1999/REC-xml-names-19990114-de.html

%Wir betrachten Anwendungen der Extensible Markup Language (XML), in denen ein einzelnes XML- Dokument Elemente und Attribute (hier "Markup-Vokabular" genannt) enthalten kann, die für verschiedene Software-Module definiert sind und von verschiedenen Software-Modulen verwendet werden. Eine Motivation dafür ist Modularität; wenn ein Markup-Vokabular existiert, das gut verstanden wird und für welches nützliche Software vorhanden ist, ist es besser, dieses Markup wieder zu verwenden als es neu zu erfinden.

%%%%%%%%%%%%%%%%%%%%%%%%%%%%%%%%%%%%%%%%%%%%%%%%5

%http://www.itwissen.info/definition/lexikon/Namensraum-XML-namespace-XML.html 17.okt 13:45

%Namensräume (engl. name spaces) sind in XML-Dokumenten ein einfaches Verfahren zur eindeutigen Benennung von Element- und Attributnamen. Dabei werden die Element- und Attributnamen so verknüpft, dass diese durch URI-Referenzen eindeutig identifiziert werden können. Diesbezüglich ist die URI-Referenz lediglich eine eindeutige Kennzeichnung, der XML-Prozessor oder auch Parser sucht anders als bei einer Referenz auf eine Document Type Definition (DTD) oder ein XML-Schema keine dort konkret hinterlegte Namensliste. 

\textbf{\huge{entity references}}

\subsection{Wohlgeformtheit}

Jedes XML-Dokument muss alle Regeln erfüllen, die in den XML-Spezifikationen beschrieben sind. So kann eine Anwendung das wohlgeformte XML-Dokument allgemeingültig verarbeiten. XML-Parser überwachen dabei die Wohlgeformtheit der XML-Dokumente. Sie lesen XML-Dokumente ein und überprüfen sie. Bei der Überprüfung werden keine external declarations verarbeitet, auch Attributwerte bekommmen keine Defaultwerte zugewiesen und werden nicht weiterverarbeitet. Der XML-Parser wird häufig von einer übergeordneten Anwendung verwendet. Findet er einen Teil des Dokuments der nicht den Spezifikationen entspricht, meldet er dies der Anwendung. Der XML-Parser überprüft das XML-Dokument weiter auf vorhandene Fehler, gibt aber der Anwendung keine Daten oder XML-Strukturen mehr weiter. Ein XML-Dokument kann weiter eingeschränkt werden, indem es validiert wird.

%java und xml d

%Wohlgeformtheit muss ausnahmslos für alle xml doks gelten. Hiermit sind bestimmte
%Regeln gemeint, die befolgt werden müssen, damit ein xml dok von einer Anwendung
%allgemeingültig verarbeitet werden kann.
%welches die in den XML-Spezifikationen beschriebenen Regeln erfüllt, bezeichnet man als wohlgeformt
%(well-formed).

%Wohlgeformtheit ist die Voraussetzung, dass eine XML-Datei verarbeitet werden kann.
%Die wohlgeformtheit eines xml doks wird von der interpretierenden Anwendung überwacht : xml parser .
%Ein xml parser kann ein xml dok einlesen und überprüfen. Er wird häufig von einer übergeordneten
%Anwendung verwendet, um Informationen aus einem xml dok zu extrahieren oder in einer anderen
%Weise zu verarbeiten. Entdeckt der xml parser einen oder mehrere Verstöße gegen die Wohlgeformtheit,
%so meldet er diese an die übergeordnete Anwendung.
%%%%%%%%%%%%%%%%%%%%%%%%%%%%%%%%%%%%%%%%%%%%%%%%%%%%%%%%%%%%%%%%%%%%%%%%%%%%%%%%%%%%%%%%%%%%%%%%
%java und xml e

%all xml data must conform to both syntax requirements and a simple container structure. 
%this data can be used without a dtd or schema to describe their structure, and is also known as standalone xml data. such data cannot rely upon any external declarations, and attribute values will receive no special processing or default values. 

%any xml parser that encounters a construct within the xml data that is not well formed must report this error to the application as a fatal error. Fatal errors need not cause the parser to terminate- it may continue processing in an attempt to find other errors, but it may not continue to pass character data and/or xml structures to the application in normal fashion. Hopefully, this rather brutal error handling will prevent the creation of bloated software like internet explorer and navigator.

\subsection{Validierung}

java und xml d

der xml parser kümmert sich beim einlesen im normalfall um die überorüfung der wohlgeformtheit des doks. ist diese verletzt schlägt der parse vorgang fehl.

zu xml dokumenten können grammatiken existieren: ddtds, xml schemas. In solchen Grammatiken sind keine syntaktischen, dafür aber inhaltliche Regeln definiert, die sich nicht darum kümmern, mit welcher schreibweise elemente, attribute, kommentare usw auszuzeichnen sind. Stattdessen wird der logische inhalt spezifiziert, also beispielswiese welche elemente und attribute in einem dok vorkommen dürfen, welche namensräume diese haben müssen und wie sie geschachtelt werden können.entspricht ein konretes xml dok den regeln aus der gramatik so nennt man es valide. 

valide: wohlgeformt, inhalt entsprichtt grammatik.
was ist validierung und wofür evtl auch hier 
%%%%%%%%%%%%%%%%%%%%%%%%%%%%%%%%%%%%%%%%%%%%%%%%%%%%%%%%%%%%%%%%%%
DTD

\textbf{entytie referenz}

Java und xml d

grammatik eines xml dok kann bis zu einem bestimmten grad vorgegeben werden. Das heißt es kann explizit vorgeschrieben werden, wie ein xml dok struktureiert sein muss und welche inhalte es besitzen darf. Dies ist beispielsweise dann äuérst wichtig, wenn eine Anwendung ein xml dok mit einer bestimmten struktur erwartet. Um zum einen nun außerhalb dieser anwendung zu dokumentiere, wie das xml dok struktueriter sein muss und andererseits der anwendung selbst eine einfache möglichkeit zur überprüfung des  eingelesenen xml dok zu geben wird die erwartete struktur fest mit hilfe eines definitionsdokuments vorgegeben: 
dtd ist am ältesten und am weitesten verbreitet. Nutzt eigene syntax die zwar an xml angelehnt aber noch nicht identisch ist. Damit ein xml dokument gegen eine grammatik überpr+ft wrefden kann, ist es erforderlich dass der verwendete xml parser die validierung anhand einer dtd bzw eines xml schemas unterstützt –> validierender parser

neben der definition der struktur eines xml doks hat eine dtd zusätzlich eine weiter wichtige funktion, nämlich entities zu definieren--> sinnvoll, emmm eie ersetzung an einer bestimmten stelle im xml dok erfolgen soll

interne und externe entities

deklaration der dt d in einer seperaten datei abspeichern und diese datei aus dem xml dok heraus zu referenzieren dtd referenzieren dtd direkt einbetten

%%%%%%%%%%%%%%%%%%%%%%%%%%%%%%%%%%%%%%%%%%%

java und xml e

dtds are a set of rules that define how xml data should be structured. Being able to define suhc rules will become more inportant as we exchange, process and display xml in a wider environwent.

Well formed xml data is guaranteed to use proper xml syntax , and a properly nested /hieraarchical ) tree structure.

Parser itself can verify that the document conforms to the rules of a specufic xml vocubalayr. This validation is accomplished by comparing the content of the dok with an associate template in the form of DTD.
 
Valid xml data is well formed data that also complies with syntax, structual, and other rules as defined in a dtd

dtds use a formal ggrammar o describe the structure and syntax of an xml dok including the permissible values for much of that dok conted. These rules, called validity constraints, ensure that any xml data conforms to its associated dtd.

A dtd is a set of declatations which can be incorporated within xml data, or exist as a seperate dok. The dtd defines the rules that describe the structure and permissible content of the mxl data. Only one dtd may be associated with a given xml structure.

The most significant aspet of dtd validation is the definition of the structure of the hierarchical tree of elements. A validating parser and a dtd can ensure that all necessary elements and attributes are present in a dok, and that there are no unauthorized elements or attributes. This ensures that the data has a valid structure before it is handed over to the application.

A dtd can be used in conjunction with a validating parser to validate existing xml data or envorcee validity during the vreation of xml doks by ahuman author Sharing dtds are the basis for many xml vocabularys. --> more reliable data exchange.

Alternativen: xml schema cml-data, xml-data-reduced

although only one dtd can be associated wiht a given xml dok, that dtd may be divided into two parts: internal und external subset

the internal subset is that portion of the dtd inicluded within the xml data (priority)

the external subset is the set of declarations that are located in a seperate documents without dtd we can‘t validate the xml data (we can at least ensure that it‘s wellformed).

Dtds (doc type definition) are linked to xml data objects using markup called dokument type declaration

validating oarsers use this declaration to retrieve the dtd (if it exists) and validate the document according to the dtds rules. If the dtd is not found, the parser will send an error message, and be unable to validate the document.

Most dtds ue the external subset, since this allows sharing the dtd with multiple doks, and seperates the xml data and its desscriptions. Using only internal subset would require copies of the dtd declarations in all documet instances. External subsets update is so auch einfacher.

Internal subsets can be used to override an external dtd, its the only way that an existing dtd that doesnt match our needs exactly can be used.

Dtd dexlarations that describe the most xml doks:

Elements

Content models (element declaration to describe the structure and content of a gien

element type

Sequenve and choice lists

Attributs

Entities

Notations -Non xml data

cml is based upon text data, but not all data is textual. We need some way to include references to external data, such as images, spreadsheet files, and other binary data formats. Thus, we describe the non xml data in the dtd and include a reference to an external applivation that will handle the data (since an xml parser cant do this). The xml applivcation will then have this info available either for its own use or to pass on to a helper application.

A notation is used by the xml application as a hint about handling an unparsed entity or some other non xml data. 

Notations can be used to identify:

the format of unparsed entities

the format of element attributes of the entity and entity types

the application associated with a processing instructions

Entities

Replaceable content is a key maintenance . References in our xml data. Reference to pre defined character entity sets. The document entity serves as the entry point for an xml parser, and contains the entire dok. Internal and external subsets are also entities.

Two main categories of entities:

general entities: used within xml data (can be parsed or unparsed)

parameter entities: only used in dtds



parsed entities: can include any well-formed content, known  as the replacement text which may include other markup

unparsed entities: non xml data

internal: the actual replacement is included in the declarations

external: the replacement text is lovated in an external file or

other ressource (public, system, or both)

no inheritance

when the xml parser encounters the external reference, it will refer back to the earlier declaration, read the named file (at the system location), process the declaration therein, and then pass its replacement text to the application, just as if the entity were declared in the dok as before.

Every validating parser must include 2 different parsere, one for xml and one for the dtd.

This parsers keep the dtd information hidden from the application and user.

If amespaces are used, every element type from each namespace must be included in the dtd.

Dtd really work with only one data type-- the text string. There is no provision for nummeric data types, or much less for more complex structures like dates, times, encoded numbers or strings, or uri reference.

The difference between well formed and valid xml data-modelthah dtds comprise an internal and an external subset how to describe elements, attributes and their content.

How to define the various kids of entities, for both xml data and dtd.

How to handle non xml data-modelhow to use conditional sections in dtds the limitations of dtds.

%%%%%%%%%%%%%%%%%%%%%%%%%%%%%%%%%%%%%%%%%%%

wikipedia

Schemasprachen, um Dokumenttypdefinitionen auszudrücken;

MitDocument Schema Definition Languages existiert eine eigene Spezifikation zur Definition von Dokumentstrukturen, Datentypen und Datenbeziehungen in strukturierten Informationsquellen.

Innerhalb einer DTD kann die Dokumentstruktur mit Deklarationen von Elementtypen, Attributlisten, Entities und Notationen und Textblöcken definiert werden. Dabei können spezielle Parameter-Entities benutzt werden, die DTD-Teile enthalten und nur innerhalb der DTD erlaubt sind.

Notationen sind Hinweise zur Interpretation von externen Daten, die nicht direkt vom XML-Parser verarbeitet werden. Notationen können sich beispielsweise auf ein Dateiformat für Bilder beziehen.

%%%%%%%%%%%%%%%%%%%%%%%%%%%%%%%%%%%%%%%%%%%

w3 school

An XML document validated against a DTD is both "Well Formed" and "Valid".  The markup declarations can be given either locally, in the root entity of the document in the document type declaration , or externally in a separate fi le, as a separate entity. In the latter case, the address of the fi le must be provided by the document type declaration.

%%%%%%%%%%%%%%%%%%%%%%%%%%%%%%%%%%%%%%%%%%%

XML: DTD, XML-Schema, XPath, XQuery, XSLT, XSL-FO, SAX, DOM 15.oktober 13:51

Wohlgeformtheit sicher nur die syntaktische Korrrektheit eines xml doks. In vielen Bereichen ist es zudem sehr wichtig sicherzustellen, dass ein Dok auch einer vorab festgelegten Struktur entspricht. Die älteste Möglichkeit, eine Strukturdefinition für eine Klasse von Doks zu definieren, ist die form einer dtd.

In einer dtd wird festgelegt, welche elemente erlaubt sind, wie diese verschachtelt sind und welche Attribute ein Element haben darf oder muss. Ein xml dok ist gültig wenn seine struktur der zugehörigen dtd entspricht. Ein gültiges dok ist immer wohlgeformt.
%%%%%%%%%%%%%%%%%%%%%%%%%%%%%%%%%%%%%%%%%%%%%%%%%%%%%%%%%%%%%%%%%%%
XML-Schema

\textbf{xinclusion}

Java und xml d

dtd ist eine relatiiv unkomplizierte möglichkeit, ein xml dok bis zu einem bestimmten grad zu beschränken und sommit für viele anwendungen bestimmt eine gute wahl.

Nachteil dtd: dtd liegt nicht in einem xml format vor, weswegen bereits für die bearbeitung häufig verschiedene Werkzeuge verwendet werden müssen.

Wenig granulare definitionsmöglichkeit von dtd. Für viele moderne anwengdungen reicht die definition von element – verscchachtelung und attribut-angaben häufig nicht weiter, indem beispielsweise eigene elementtypen definiert und vererbt werden können und sich wesentlich mehr einschränkungen für die werte von Attribut und Elementinhalten festlegen lassen.

Xml schema kann wie dtd dok beschränken --> in form von einem weiteren xml dok. Die beschränkung selbst ann auf einem wesentlich feineren Level erfolgen. Allerdings steigt auch der kompläxitätsgrad an.

Ein xml schema ist im bezug auf die beschränkung eines xml doks zwar wesenlich mächtiger als eine dtd, allerdings ist für manche einsatzzwecke trotz allem parallell zu xml schema auch eine dtd notwendig --> dtd kann entity referenzen definieren, xml schema nicht.

Ausgangspunkt einer sml schema grammatik sind typen (einfache und komplexe) 

einfache typen: alles was sich in xml durch reinen text ohne tags ausdrücken lässt und damit in attributen oder als textknoten in elementen unterscheidungen möglich. (int, boolean, decimal, time Wertelisten, unions).

Um einen eigenen einfachen typ zu dfinieren , muss ein element in die schemadatei eingafügt werden.

Komplexe typen diehneen dazu inhalte zu definieren, die auch anderes als reinen text enthalten können, beispielsweise kindelemente oder attribute. Nur elementen kann ein omplexer typ zugewiesen werden, während einfachhe typen sowohl elemente als auch attributen zugewiesen werden.

Xml schema unterscheidet bei komplexen typen zwischen einfachem und komplexem inhalt.

Einfaccher: element darf nur text und attribute und keine kindelemente enthalten

Komplexer: Kindelemente erlaubt, textinhalt erlaubt- kann aber keinen regeln unterworfen werden

%%%%%%%%%%%%%%%%%%%%%%%%%%%%%%%%%%%%%%%%%%%

wikipedia

Struktur in Form eines XML-Dokuments beschrieben

große Anzahl von Datentypen unterstützt.

Im Gegensatz zu DTDskann bei Verwendung von XML Schemata zwischen dem Namen des XML- Typs und dem in der Instanz verwendeten Namen des XML-Tags unterschieden werden.

Vergleichbar den Primärschlüsseln in relationalen Datenbanken lassen sich mittels XML Schema eindeutige Schlüssel definieren. XML Schema unterscheidet zwischen der Eindeutigkeit (engl. unique) und der Schlüsseleigenschaft.

%%%%%%%%%%%%%%%%%%%%%%%%%%%%%%%%%%%%%%%%

w3 school

XML-based alternative to DTD

XML Schemas are written in XML

• XML Schemas are extensible to additions

• XML Schemas support data types

• XML Schemas support namespaces

%%%%%%%%%%%%%%%%%%%%%%%%%%%%%%%%%%%%%%%%%%

https://www.w3.org/TR/xmlschema11-1/

The schema language, which is itself represented in an XML vocabulary and uses namespaces, substantially reconstructs and considerably extends the capabilities found in XML document type definitions (DTDs). The purpose of an XSD schema is to define and describe a class of XML documents by using schema components to constrain and document the meaning, usage and relationships of their constituent parts In general, a valid document is a document whose contents obey the constraints expressed in a particular schema

%%%%%%%%%%%%%%%%%%%%%%%%%%%%%%%%%%%%%

java und xml e

It is a powerful and flexible language, which permits authors to create either tighter or looser restrictions over doks than dtd, while still allowing them to be validated. Schemas describe the structure of doks and constrain what they can contain.

%%%%%%%%%%%%%%%%%%%%%%%%%%%%%%%%%%

http://www.aspheute.com/artikel/20010514.htm 15.oktober 10:44

Die DTD Spezifikationen wurden vom W3C Konsortium vorgeschlagen, während die XML Schemas auf einen Vorschlag von Microsoft basieren aber jetzt vom W3C Konsortium vereinheitlicht und standardisiert wurden.

Diese Informationen benötigen andere Applikationen (z.B. Web Services), damit diese erkennen, in welcher Form bzw. Datenstruktur sie die Daten erhalten, wie sie die erhaltenen Daten bearbeiten müssen und in welcher Form sie sie zurück zu Ihnen senden müssen, damit Sie wiederum mit den bearbeiteten Ergebnissen etwas anfangen können.

Weiters können Sie Elementen einen Datentyp zuweisen - z.B Integer, Datumswert, Fließkommazahlen, Zeichenketten sowie die anderen (simplen) Datentypen oder auch URLs. Bei der Verwendung von DTD's sind Sie bei der Datentypzuweisung auf den Datentyp String limitiert.

XML Schemas ist die einfache Erweiterbarkeit dieser, da sie XML basiert sind und daher nicht an einen vorgegebenen Syntax gebunden sind

referenziert eine XML Datei dieses Schema, und die XML Datei entspricht dem Schema nicht, dann wird der Parser die XML Datei als nicht korrekt monieren.

%%%%%%%%%%%%%%%%%%%%%%%%%%%%%%%%%%%%%%%%%%%

http://www.uni-weimar.de/medien/webis/teaching/lecturenotes/web-technology/unit-de-
doclang-xml-schema.pdf 15.oktober 10:52

Das Typsystem von XML-Schema ermöglicht: 
Definition von Constraints für zugelassenen Inhalt 
Überprüfung der Korrektheit von Daten 
Verarbeitung von Daten aus Datenbanken 
Spezialisierung von Datentypen 
Definition komplexer Datentypen 
Konvertierung zwischen Daten verschiedenen Typs

Zwei Arten von Namensräumen, die deklariert werden können [W3C] : 1.

Einen oder mehrere Namensräume, zu denen die im Schema verwendeten Elemente und Datentypen (= das XSD-Vokabular) gehören. 2. Einen Zielnamensraum (Target Namespace) für die global deklarierten Elemente und Attribute des Schemas (= das Autorenvokabular). Dieser muss bei schemakonformen Instanzdokumenten beachtet werden.

Ein XML-Dokument heißt gültig hinsichtlich eines Schemas (schema valid), wenn es über ein Schema verfügt, und konform zu diesem aufgebaut ist. Das zu validierende Dokument wird als Instanzdokument bezeichnet. Dadurch wird es möglich, alle erlaubten Schemata selbst durch ein Schema – ein sogenanntes Metaschema – zu beschreiben und zu validieren
%%%%%%%%%%%%%%%%%%%%%%%%%%%%%%%%%%%%%%%%%%%%%%%%%%%%%%%%%%%%%%%%%%%
Andere

\subsection{XSL -- Extensible Stylestheet Language}
\subsubsection{Navigation}

Xpath – XML Path Language B

\textbf{Factories? WDDX, XQuery, XPAth in DOM SAX,XPAth data model , xPath injection}
XPath ist eine Sprache, mit derer Hilfe in XML-Dokumenten navigiert werden kann. Dazu erstellt sie aus dem Infoset des XML-Dokuments ein Datenmodell. In diesem sind die einzelnen XML-Bausteine als  Knoten eines hierarchischen XML-Baumes realisiert. Es gibt also zum Beispiel einen Wurzelknoten, Elementknoten oder Textknoten. Jeder Knoten wird durch einen erweiterten Namen, bestehend aus einem lokalem Teil in Form eines Strings und einem Namensraum  identifiziert. Für Elemente desselben XML-Dokuments besteht anhand der Start Tags eine bestimmte Ordnung in der sie im XML-Baum auftreten. Die Reihenfolge von Attributen, Namespaceknoten und von Knoten aus unterschiedlichen Quellen ist aber nicht  ausschließlich festgelegt. Somit ist der Aufbau des XML-Baumes nicht vollständig definiert. 
Mittels XPath können dann einzelne Knoten oder ganze Knotenmengen ermittelt werden und deren Informationen extrahiert werden. Dazu wird eine Syntax verwendet, die nicht auf XML aufbaut. So kann XPath einfach verwendet werden um in URIs oder XML-Attributswerten Information zu finden.  XPath stellt auch Standardfunktionen zur Manipulation von Strings, Zahlen, Booleans und Knotenmengen bereit. Mit ihrer Hilfe können die Knoten genauer eingeschränkt und lokalisiert werden. Dafür benutzt XPath einen Lokalisierungspfad.
Ein Lokalisierungspfad wird verwendet um vom Wurzelelement zu einem bestimmten Knoten oder einer bestimmten Knotenmenge innerhalb eines XML-Dokuments zu navigieren. Ein Lokalisierungspfad setzt sich aus mehreren Lokalisierungsschritten zusammen. In jedem Lokalisierungsschritt werden Angaben bezüglich der Axe, dem Knotentyp und den Einschränkungen, denen die Knotenmenge unterliegen soll, benötigt. 
Mithilfe der \textbf{Axe} wird festgelegt in welche Richtung der Schritt gehen soll. Beispiele für Axenrichtungen sind \texttt{child}, \texttt{attribute}, \texttt{descendant}, \texttt{ancestor} oder \texttt{self}.
Der \Textbf{Knotentyp} und der erweiterte Name des Knotens werden mithilfe eines Knotentests spezifiziert. Knotentypen können hierbei zum Beispiel Attribute, Namespaces oder Elemente sein.
Die bisher gefundene Knotenmenge kann dann durch \textbf{Einschränkungen} begrenzt werden.  Diese Einschränkungen können mithilfe von Null oder mehreren boolschen Prädikaten erfolgen.
XPath kommt in mehreren anderen XML basierten Technologien zum Einsatz. Beispiele sind XSLT, XPointer, XForms oder XML Schema.A


%%%%%%%%%%%%%%%%%%%%%%%%%%%%%%%%%%%%%%%%%%%%%%%%%%%%%%%%%%%%%%%%%%%%%%%%%%%%%%%%%%%%%%%%%%%%%%%%%%%%%%%%%

\textbf{Factories? WDDX, XQuery, XPAth in DOM SAX,XPAth data model , xPath injection}

XPath (XML Path Language) ist eine Sprache die die Addressierung von, oder die Navigation zu ausgewählten Teilen von XML-Dokumenten erlaubt. Es wird eine nicht-XML Syntax verwendet, um den Einsatz von URIs und XML Attributswerten zu vereinfachen. Standard Funtionen zur Manipulation von Strings, Zahlen, Booleans und Knotenmengen werden bereitgestellt. B

XPath erstellt aus dem Infoset des XML-Dokuments ein Datenmodell, welches die Knoten als hierarchischen XML-Baum modelliert. Es gibt einen Wurzel Knoten, Element Knoten, Text Knoten, Attribut Knoten, Namespace Knoten, Processing Instruction Knoten und Kommentar Knoten. Jeder Knoten hat einen erweiterten Namen, bestehend aus einem lokalen Teil (String) und einem Namensraum (URI oder NULL). Für Elemente desselben Xml-- Dokuments besteht anhand der Start Tags eine bestimmte Ordnung. Die Reihenfolge von Attributen, Namespace Knoten und von Knoten aus unterschiedlichen Quellen ist nicht vollständig definiert. B

Mithilfe eines Lokalisierungspfades wird eine Menge von Knoten zurückgegeben. Er besteht aus einzelnen Lokationsschritten. Jeder einzelne Schritt besteht aus drei Angaben: Der Axe, in welche Richtung der Schritt geht (z. B.: Self, Child, Parent, Ancestor). Dem Knotentest, der den Knoten Typ und den erweiterten Namen spezifiziert. Und Null oder mehreren boolschen Prädikaten, die die bisher gefundene Knotenmenge einschränken.B

Xpath ist in mehreren anderen XML basierten Technologien, zum Beispiel in XSLT, XPointer, XForms oder dem XML Schema, integriert. A

%has a set of functions that provide access to XML documents (fn:doc, fn:doc-available), collections (fn:collection, fn:uri-collection), text files (fn:unparsed-text, fn:unparsed-text-lines, fn:unparsed-text-available), and environment variables (fn:environment-variable, fn:available-environment-variables). These functions are defined in Section 14.6 Functions giving access to external information FO31. D

%Java und xml d		A

%java und xml e		B

%w3c				C

%w3c path 3.1		D



\subsubsection{Transformation}

\textbf{rdf, pop und mom als anwendungsgebiete}

XSLT (Extensible Stylesheet Language for Transformations) ist eine Sprache, die ein XML-Dokument in ein anderes Textdokument oder wieder in ein XML Dokument umwandeln kann.

Für eine Transformation werden ein Quelldokument (XML-Dokument) und ein oder mehrere XSLT Stylesheets benötigt. Ein Stylesheet besteht dabei aus Templates, diese spezifizieren wie das Quelldokument transformiert werden soll. Ein Template besteht aus einem Xpath Ausdruck und den zugehörigen Regeln. Der Xpath ausdruck ermittelt die XML Elemente, die dann den Regeln entsprechend transformiert werden. A

Der XSLT Prozessor erhält das Quelldokument und den Stylesheet. Danach erstellt er auf Basis des Quelldokuments wieder ein Datenmodell eines hierarchischen XML-Baumes. Der Stylesheet wird auf das Quelldokument angewendet und es wird ein Ergebnisbaum erstellt. Dieser wird dann in ein Zieldokument umgewandelt, ein XML Dokument, ein HTML Dokument oder ein anderes Textformat. B

%Java und xml d		A
%java und xml e		B


\subsubsection{Formatierung}

\input{XSLFO}

\subsection{Verlinkungen}

Fragments?

X-Link

X-Pointer

\subsection {XML-Parser APIs}

DOM

SAX

Stax

Bindings

\subsection{Sicherheitsstandards}

XML Digital Signature (XML DigSig)

XML Encryption (XML Enc)

XML Signature Proeprties




XML Key Management Specifification (XKMS)

Security Assertion Markup Language (SAML)

XML Access Control Markup Language (XACML)

eXtensible rights Markup Language (XrML)

Platform for Privacy Preferences (P3P)
 
