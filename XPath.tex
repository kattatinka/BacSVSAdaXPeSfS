\textbf{Factories? WDDX, XQuery, XPAth in DOM SAX,XPAth data model , xPath injection}

XPath (XML Path Language) ist eine Sprache die die Addressierung von, oder die Navigation zu ausgewählten Teilen von XML-Dokumenten erlaubt. Es wird eine nicht-XML Syntax verwendet, um den Einsatz von URIs und XML Attributswerten zu vereinfachen. Standard Funtionen zur Manipulation von Strings, Zahlen, Booleans und Knotenmengen werden bereitgestellt. B

XPath erstellt aus dem Infoset des XML-Dokuments ein Datenmodell, welches die Knoten als hierarchischen XML-Baum modelliert. Es gibt einen Wurzel Knoten, Element Knoten, Text Knoten, Attribut Knoten, Namespace Knoten, Processing Instruction Knoten und Kommentar Knoten. Jeder Knoten hat einen erweiterten Namen, bestehend aus einem lokalen Teil (String) und einem Namensraum (URI oder NULL). Für Elemente desselben Xml-- Dokuments besteht anhand der Start Tags eine bestimmte Ordnung. Die Reihenfolge von Attributen, Namespace Knoten und von Knoten aus unterschiedlichen Quellen ist nicht vollständig definiert. B

Mithilfe eines Lokalisierungspfades wird eine Menge von Knoten zurückgegeben. Er besteht aus einzelnen Lokationsschritten. Jeder einzelne Schritt besteht aus drei Angaben: Der Axe, in welche Richtung der Schritt geht (z. B.: Self, Child, Parent, Ancestor). Dem Knotentest, der den Knoten Typ und den erweiterten Namen spezifiziert. Und Null oder mehreren boolschen Prädikaten, die die bisher gefundene Knotenmenge einschränken.B

Xpath ist in mehreren anderen XML basierten Technologien, zum Beispiel in XSLT, XPointer, XForms oder dem XML Schema, integriert. A

%has a set of functions that provide access to XML documents (fn:doc, fn:doc-available), collections (fn:collection, fn:uri-collection), text files (fn:unparsed-text, fn:unparsed-text-lines, fn:unparsed-text-available), and environment variables (fn:environment-variable, fn:available-environment-variables). These functions are defined in Section 14.6 Functions giving access to external information FO31. D

%Java und xml d		A

%java und xml e		B

%w3c				C

%w3c path 3.1		D

