

Xslt
\textbf{rdf, pop und mom als anwendungsgebiete}

XSLT (Extensible Stylesheet Language for Transformations) ist eine Sprache, die derer Hilfe ein XML-Dokument verändert und transformiert werden kann. 
Für die Transformation werden ein Quelldokument in Form eines XML-Dokuments und ein oder mehrere XSLT-Stylesheets benötigt. Ein Stylesheet besteht aus Templates die spezifizieren wie das Quelldokument transformiert werden soll. Ein Tamplate besteht aus einem XPath Ausdruck und den zugehörigen Regeln. Der XPath Ausdruck ermittelt dann die XML Elemente, die den Regeln entsprechen transformiert werden. A
Die Transformation des XML-Dokuments wird von einem XSLT-Prozessor durchgeführt. Er erhält das Quelldokument und das Stylesheet. Er erstellt auf Basis des XML-Dokuments wie XPath das Datenmodell in Form eines hierarchischen XML-Baumes. Der Stylesheet wird auf diesen angewendet und es wird ein Ergebnisbaum erstellt. Der Ergebnisbaum wird dann in ein Zieldokument, in Form eines XML-Dokuments, eines HTML-Dokuments oder eines anderen Textformats, umgewandelt und zurückgegeben. B



%%%%%%%%%%%%%%%%%%%%%%%%%%%%%%%%%%%%%%%%%%%%%%%%%%%%%%%%%%%%%%%%%%%%%%%%%%%%%%%%%%%%%%%%%%%%%%%%%%%%%

\textbf{rdf, pop und mom als anwendungsgebiete}

XSLT (Extensible Stylesheet Language for Transformations) ist eine Sprache, die ein XML-Dokument in ein anderes Textdokument oder wieder in ein XML Dokument umwandeln kann.

Für eine Transformation werden ein Quelldokument (XML-Dokument) und ein oder mehrere XSLT Stylesheets benötigt. Ein Stylesheet besteht dabei aus Templates, diese spezifizieren wie das Quelldokument transformiert werden soll. Ein Template besteht aus einem Xpath Ausdruck und den zugehörigen Regeln. Der Xpath ausdruck ermittelt die XML Elemente, die dann den Regeln entsprechend transformiert werden. A

Der XSLT Prozessor erhält das Quelldokument und den Stylesheet. Danach erstellt er auf Basis des Quelldokuments wieder ein Datenmodell eines hierarchischen XML-Baumes. Der Stylesheet wird auf das Quelldokument angewendet und es wird ein Ergebnisbaum erstellt. Dieser wird dann in ein Zieldokument umgewandelt, ein XML Dokument, ein HTML Dokument oder ein anderes Textformat. B

%Java und xml d		A
%java und xml e		B
